\documentclass{beamer}

\usepackage[utf8]{inputenc}
\usepackage{listings}
\usepackage{color}
\usepackage{enumitem}
\usepackage{hyperref}
\usepackage[orientation=landscape,size=custom,width=16,height=9,scale=0.5,debug]{beamerposter}

\usepackage[ngerman]{babel}
\usepackage[ngerman]{isodate}
\usepackage[parfill]{parskip}


\usepackage{ltablex} % mix out of tabularx and longtable
\usepackage{multirow}

\usepackage{graphicx}
% \usepackage{wrapfig}
\usepackage{subcaption} %To create subfigures
% \usepackage{subfig} %To create subfigures
\usepackage{placeins} % FloatBarrier
% \usepackage{floatrow}
\graphicspath{ {./images/} }

\usepackage[]{geometry}
\usepackage{pdflscape}
\usepackage{booktabs}

\usepackage[most]{tcolorbox}
\usepackage{cleveref}
\usepackage{tikz}
\usetikzlibrary{decorations.pathreplacing,shapes,arrows,positioning}
\usetikzlibrary{positioning}
\usetikzlibrary{backgrounds}
\usetikzlibrary{patterns}
\usetikzlibrary{calc}
\usetikzlibrary{fit}

\tikzstyle{input} = [coordinate]
\tikzstyle{output} = [coordinate]
\tikzstyle{block} = [rectangle, draw, text width=5em, text centered,  minimum height=4em]
\tikzstyle{storage} = [cylinder, shape border rotate=90, aspect=0.25, draw]
\tikzstyle{label} = [text width=2.4cm, text centered]


\definecolor{dkgreen}{rgb}{0,0.6,0}
\definecolor{gray}{rgb}{0.5,0.5,0.5}
\definecolor{mauve}{rgb}{0.58,0,0.82}

\lstset{frame=tb,
  language=C++,
  aboveskip=3mm,
  belowskip=3mm,
  showstringspaces=false,
  columns=flexible,
  basicstyle={\small\ttfamily},
  numbers=left,
  numberstyle=\tiny\color{gray},
  keywordstyle=\color{blue},
  morekeywords={vector},
  commentstyle=\color{dkgreen},
  stringstyle=\color{mauve},
  breaklines=true,
  breakatwhitespace=true,
  tabsize=3
}



\definecolor{lmugreen}{RGB}{50,55,44}

\setbeamercolor{title}{fg=lmugreen}
\setbeamercolor{titlelike}{fg=lmugreen}
%\setbeamertemplate{itemize items}[circle]
\setbeamertemplate{itemize items}{\color{black}$\blacktriangleright$}
%\setbeamertemplate{footline}[frame number]

\setbeamertemplate{footline}[text line]{%
  \parbox{\linewidth}{\vspace*{-8pt}c3particles\hfill\insertshortauthor\hfill\insertframenumber/\inserttotalframenumber}}
\setbeamertemplate{navigation symbols}{}


%Information to be included in the title page:
\title{\textbf{Cimarron}\\Stabilisation of videos in modern \texttt{C++}}
\subtitle{Praktikumsabschlusspr\"asentation}
\author{Marius Herget}
\date{\today}
\institute{Institut f\"ur Informatik, LMU M\"unchen}

\usebackgroundtemplate%
{%
    \includegraphics[width=\paperwidth,height=\paperheight]{images/bg-empty.pdf}%
}
\setbeamertemplate{frametitle}[default][center]

\begin{document}

\frame{\titlepage}

\begin{frame}
\frametitle{Zielsetzung}
\begin{description}
    \item[Videostabilisierung] Methodik um ungewollte Bewegungen in Video zu reduzieren
    \item[\texttt{C++}] eignet sich besonders gut, da formale Konzept logisch implementiert und der Speicher effizient verwaltet werden kann
    \item[Ziel] Entwicklung einer Programmierabstraktion zum Kompensieren von ungewollten Bewegungen (Translation und Rotation) mit Hilfe von Feature Tracking
\end{description}
\end{frame}

\begin{frame}

\frametitle{Bewegungsmuster}
\begin{figure}\centering
    \begin{minipage}{.45\textwidth}\centering
        \begin{tikzpicture}[scale=1]
        \draw[step=0.5cm,gray,thin] (-3,-2) grid (2,1);
        \draw[black, fill = black] (-2,-0.5) circle [radius=.5];
        \draw[black, pattern=dots] (1,-0.5) circle [radius=.5];
        \draw[thick, black, ->, line width=1mm] (-1.25,-0.5) -- node[above]{\small$\overrightarrow{v} = (6,0)$} (0.25,-0.5);
        \end{tikzpicture}
        \subcaption{Translational motion}
    \end{minipage}
    \begin{minipage}{.45\textwidth}\centering
        \begin{tikzpicture}[scale=1]
        \draw[step=0.5cm,gray,thin] (-3,-2) grid (2,1);

        \draw[black, fill = black]  (-2.5,0) rectangle (-1.5,-1);
        \draw[black, pattern=dots,rotate around={45:(1,-0.5)}] (0.5,0) rectangle (1.5,-1);
        % \draw[thick, black, ->, line width=1mm] (-1.25,-0.5) -- (0.15,-0.5);
        \draw[-stealth,  black, line width=0.5mm] (-1.75,0.25) arc  (100:0:0.5)node[above right]{\small$45^\circ$};
        \end{tikzpicture}
        \subcaption{Rotational motion}
    \end{minipage}
    \caption{Differnt motion models}
    \label{fig:motionmodels}
\end{figure}
\end{frame}

\begin{frame}
\frametitle{Stabilisierung Theorie}
\begin{figure}\centering
    \begin{minipage}{.45\textwidth}\centering
        \begin{tikzpicture}[scale=1]
        \draw[step=0.5cm,gray,thin] (-3,-2) grid (2,1);
        \draw[black, fill = black] (-2,-0.5) circle [radius=.5];
        \draw[black, pattern=dots] (1,-0.5) circle [radius=.5];
        \draw[thick, black, ->, line width=1mm] (-1.25,-0.5) -- node[above]{\small$\overrightarrow{v} = (6,0)$} (0.25,-0.5);
        \draw[thick, white, ->, line width=0.001mm] (-1.25,1.1) -- node[above]{\small$\overrightarrow{v} = (6,0)$} (-2.75,1.1);
        \end{tikzpicture}
        \subcaption{Erkennung}
    \end{minipage}
    \begin{minipage}{.45\textwidth}\centering
        \begin{tikzpicture}[scale=1]
        \draw[step=0.5cm,gray,thin] (-3,-2) grid (2,1);
        \draw[black, fill = gray] (-2,-0.5) circle [radius=.5];
        \draw[thick, blue, ->, line width=0.5mm] (-1.25,1.1) -- node[above]{\small$\overrightarrow{v} = (-6,0)$} (-2.75,1.1);

        \draw[step=0.5cm,blue,thin] (-5,-2) grid (-1,1);
        \draw[blue, pattern=dots] (-2,-0.5) circle [radius=.5];


        \end{tikzpicture}
        \subcaption{Gegenmassnahme}
    \end{minipage}
    \caption{Translationsstabilisierung}
    \label{fig:motionmodels}
\end{figure}
\end{frame}
%
\begin{frame}
\frametitle{Stabilisierung Theorie}
\begin{figure}\centering
    \begin{minipage}{.4\textwidth}\centering
        \begin{tikzpicture}[scale=1]
        \draw[step=0.5cm,gray,thin] (-3,-2) grid (2,1);

        \draw[black, fill = black]  (-2.5,0) rectangle (-1.5,-1);
        \draw[black, pattern=dots,rotate around={45:(1,-0.5)}] (0.5,0) rectangle (1.5,-1);
        % \draw[thick, black, ->, line width=1mm] (-1.25,-0.5) -- (0.15,-0.5);
        \draw[-stealth,  black, line width=0.5mm] (-1.75,0.25) arc  (100:0:0.5)node[above right]{\small$45^\circ$};
        \end{tikzpicture}
        \subcaption{Erkennung}
    \end{minipage}
    \begin{minipage}{.58\textwidth}\centering
        \begin{tikzpicture}[scale=1]
        \draw[step=0.5cm,gray,thin] (-3,-2) grid (2,1);
        \draw[step=0.5cm,blue,rotate around={45:(1,-0.5)},thin] (-3,-2) grid (2,1);

        \draw[black, fill = black]  (-2.5,0) rectangle (-1.5,-1);
        \draw[blue, pattern=dots,] (0.5,0) rectangle (1.5,-1);
        % \draw[thick, black, ->, line width=1mm] (-1.25,-0.5) -- (0.15,-0.5);
        \draw[-stealth,  blue, line width=0.5mm] (-2.7,-1.8) arc  (80:275:0.5)node[below left]{\small$-45^\circ$};
        \end{tikzpicture}
        \subcaption{Gegenmassnahme}
    \end{minipage}
    \caption{Rotationsstabilisierung}
    \label{fig:motionmodels}
\end{figure}
\end{frame}


\begin{frame}
\frametitle{Ähnlichkeit}
\begin{description}
    \item[Vektoren] Cosine similarity
    \[cos\_sim = \frac{A \cdot B}{\Vert A\Vert \Vert B\Vert} = \frac{\sum\limits_{i=1}^{n} A_iB_i}{\sqrt{\sum\limits_{i=1}^{n}A_i^2} \sqrt{\sum\limits_{i=1}^{n}B_i^2}}\]
    \item[Winkel] Prozentuale Veraenderung
\end{description}
\end{frame}
\end{document}
