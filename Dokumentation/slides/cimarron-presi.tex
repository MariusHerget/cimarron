\documentclass{beamer}

\usepackage[utf8]{inputenc}
\usepackage{listings}
\usepackage{color}
\usepackage{enumitem}
\usepackage{hyperref}
\usepackage[orientation=landscape,size=custom,width=16,height=9,scale=0.5,debug]{beamerposter}

\usepackage[ngerman]{babel}
\usepackage[ngerman]{isodate}
\usepackage[parfill]{parskip}



\usepackage[natbib=true,style=authoryear,backend=bibtex,useprefix=true]{biblatex}
\setbeamercolor*{bibliography entry title}{fg=black}
\setbeamercolor*{bibliography entry location}{fg=black}
\setbeamercolor*{bibliography entry note}{fg=black}
\setbeamertemplate{bibliography item}{}
\renewcommand*{\bibfont}{\scriptsize}
\addbibresource{../lib/research.bib}

\usepackage{ltablex} % mix out of tabularx and longtable
\usepackage{multirow}

\usepackage{graphicx}
\usepackage{wrapfig}
\usepackage{subcaption} %To create subfigures
% \usepackage{subfig} %To create subfigures
\usepackage{placeins} % FloatBarrier
% \usepackage{floatrow}
\graphicspath{ {../images/} }

\usepackage[]{geometry}
\usepackage{pdflscape}
\usepackage{booktabs}

\usepackage[most]{tcolorbox}
\usepackage{cleveref}
\usepackage{tikz}
\usetikzlibrary{decorations.pathreplacing,shapes,arrows,positioning}
\usetikzlibrary{positioning}
\usetikzlibrary{backgrounds}
\usetikzlibrary{patterns}
\usetikzlibrary{calc}
\usetikzlibrary{fit}
\usepackage[absolute,overlay]{textpos}

\tikzstyle{input} = [coordinate]
\tikzstyle{output} = [coordinate]
\tikzstyle{block} = [rectangle, draw, text width=5em, text centered,  minimum height=4em]
\tikzstyle{storage} = [cylinder, shape border rotate=90, aspect=0.25, draw]
\tikzstyle{label} = [text width=2.4cm, text centered]
\tikzstyle{wideblock} = [rectangle, draw, text width=7em, text centered,  minimum height=4em]


\definecolor{dkgreen}{rgb}{0,0.6,0}
\definecolor{gray}{rgb}{0.5,0.5,0.5}
\definecolor{mauve}{rgb}{0.58,0,0.82}

\lstset{frame=tb,
  language=C++,
  aboveskip=3mm,
  belowskip=3mm,
  showstringspaces=false,
  columns=flexible,
  basicstyle={\small\ttfamily},
  numbers=left,
  numberstyle=\tiny\color{gray},
  keywordstyle=\color{blue},
  morekeywords={vector},
  commentstyle=\color{dkgreen},
  stringstyle=\color{mauve},
  breaklines=true,
  breakatwhitespace=true,
  tabsize=3
}



\definecolor{lmugreen}{RGB}{50,55,44}

\setbeamercolor{title}{fg=lmugreen}
\setbeamercolor{titlelike}{fg=lmugreen}
%\setbeamertemplate{itemize items}[circle]
% \setbeamertemplate{itemize items}{\color{black}$\blacktriangleright$}
%\setbeamertemplate{footline}[frame number]
\setbeamercolor{description item}{fg=black!80!black}
\setbeamerfont{description item}{size=\footnotesize}

\setbeamertemplate{footline}[text line]{%
  \parbox{\linewidth}{\vspace*{-8pt}cimarron\hfill\insertshortauthor\hfill\insertframenumber/\inserttotalframenumber}}
\setbeamertemplate{navigation symbols}{}


%Information to be included in the title page:
\title{\textbf{Cimarron}\\Stabilization of videos in modern \texttt{C++}}
\subtitle{Praktikumsabschlusspr\"asentation}
\author{Marius Herget}
\date{\today}
\institute{Institut f\"ur Informatik, LMU M\"unchen}

\usebackgroundtemplate%
{%
    \includegraphics[width=\paperwidth,height=\paperheight]{images/bg-empty.pdf}%
}
\setbeamertemplate{frametitle}[default][center]

\newcommand{\currentSD}{}

\begin{document}

\frame{\titlepage}

\begin{frame}
\frametitle{Zielsetzung}
\begin{description}
    \item[Videostabilisierung] Methodik um ungewollte Bewegungen (Wackler) der Kamera in einem Video zu reduzieren
    \item[\texttt{C++}] eignet sich besonders gut, da Konzepte formuliert werden koennen, auf deren Grundlage eine Anwendung implementiert werden kann und die Speicherverwaltung deterministisch und minimal ist.
    \item[Ziel] Entwicklung einer Programmierabstraktion zum Kompensieren von ungewollten Bewegungen (Translation und Rotation) der Kamera mit Hilfe von Feature Tracking
\end{description}
\end{frame}

\begin{frame}
\begin{center}
    \textbf{\huge DEMO}
\end{center}
\end{frame}

\begin{frame}
    \frametitle{System Diagram}
    \begin{figure}[h!]
        \resizebox{\textwidth}{!}{%
        
\begin{figure}[h!]
% \centered\vspace{-4.5cm}\vspace{-4.5cm}
\small\centering\vspace{-0.5cm}
\begin{tikzpicture}[node distance=2cm and 2.55cm,auto]\centering
    \node [input, name=input] {};
    \node [block, right= of input] (pre1) {Decode};
    \node [block, right= of pre1] (analysis1) {Track};
    \node [block, right= of analysis1] (analysis2) {Compare};
    \node [block, right= of analysis2] (analysis3) {FilterErrors};
    \node [block, below= of analysis3] (analysis4) {Aggregate};
    \node [block, left= of analysis4] (stabi) {Transform \textit{(Stabilize)}};
    \node [block, left= of stabi] (post1) {Reframe};
    \node [block, left= of post1] (post2) {Encode};
    \node [input, left=of post2] (output) {t};

    % \node [storage, below right=0.5cm and 0.7cm of analysis] (analysisstorage) {DB};

     \draw[->](input) -- node {Video}(pre1);
     \draw[->](pre1) -- node[label] {Frames} node[below]{$f(n)$}(analysis1);
     \draw[->](analysis1) -- node[label]{Motiondata} node[below]{$m(n)$}(analysis2);
     \draw[->](analysis2) -- node[label]{Delta Tracking} node[below]{$d(n, n+1)$}(analysis3);
     \draw[->](analysis3) -- node[label,left]{Delta Tracking} node[right]{$d(n, n+1)$}(analysis4);
     \draw[->](analysis4) -- node[label,above]{Global Delta Tracking} node[below]{$gd(n, n+1)$}(stabi);
     \draw[->](stabi) -- node[label,above]{Frames} node[below]{$f(n)$}(post1);
     \draw[->](post1) -- node[label,above]{Frames} node[below]{$f(n)$}(post2);


     \draw[->](post2) -- node[above]{Video}(output);
\end{tikzpicture}
\caption{High-level system diagram}
\end{figure}

\tikzstyle{wideblock} = [rectangle, draw, text width=7em, text centered,  minimum height=4em]
\begin{figure}\centering\vspace{-4.5cm}
    \tcbox[enhanced, sharp corners, boxsep=1mm, boxrule=1mm,colframe=gray!10!white,
        colback=white,
        borderline={2pt}{-1pt}{gray,dotted}]
        {\begin{tikzpicture}[scale=0.6,node distance=1cm and 2.5cm,auto]
            \node [input, name=input] {};
            \node [block, right= of input] (pre1) {Initialize tracking areas};
            \node [wideblock, right= of pre1] (pre2) {Histogram \& Backprojection};
            \node [block, right= of pre2] (pre3) {Normali-sation};
            \node [block, below= of pre3] (pre4) {Mean-Shift};
            \node [wideblock, left= of pre4] (pre5) {Adjust size of tacking area and rotation};
            \node [input, left=of pre5] (output) {t};

            \draw[->](input) -- node[label]{Frames}node[below]{$f(n)$}(pre1);
            \draw[->](pre1) -- node[label]{Frames}node[below]{$f(n)$}(pre2);


            \draw[->](pre2) -- node[label]{$bj(f(n))$}node[below]{$hsv(f(n))$}(pre3);

            \draw[->](pre3) -- node[label, left]{$bj(f(n))$}node[right]{$hsv(f(n))$}(pre4);


            \draw[->](pre4) -- node[label, above]{Tackwindow}node[below]{$tv(f(n), ta)$}(pre5);
            \draw[->](pre5)  |- ++(0,-3cm) --node[label, above]{Tackwindow}node[below]{$tv(f(n), ta)$} ++(8.8cm,0) -- (pre4);


            \draw[->](pre5) -- node[label, above]{Motiondata}node[below]{$m(n)$}(output);
        \end{tikzpicture}
        }
    \caption{Track: CamShift algorithms}
    \label{fig:motionmodels}
\end{figure}
}
        \caption{High-level system diagram}
    \end{figure}
\end{frame}

\begin{frame}
\frametitle{Feature Tracking}
    \begin{columns}
    \begin{column}{0.3\textwidth}
        \begin{figure}
            \includegraphics[scale=0.125]{tracking-example.jpg}
            \caption{Beispiel}
        \end{figure}
    \end{column}
    \begin{column}{0.67\textwidth}
            \textbf{CAMshift} Algorithmus von OpenCV \cite{OpenCVMe72:online}:\\[0.7em]
            % \begin{enumerate}
                \textbf{1.} Meanshift\\
                \textbf{2.} Fenstergroesse und Orientierung anpassen\\
                \textbf{3.} Wiederhole 1. bis gewuenschte Genauigkeit erreicht ist
            % \end{enumerate}
    \end{column}
    \end{columns}
    % SMALL SD
    \renewcommand{\currentSD}{\draw[red,thick,dotted] ($(analysis1.north west)+(-0.2,0.2)$)  rectangle ($(analysis1.south east)+(0.2,-0.2)$);}
    

\begin{textblock*}{\the\textwidth}(30pt,215pt)\centering
\resizebox{\textwidth}{!}{%
\begin{tikzpicture}[node distance=2cm and 2.55cm,auto]\centering
    \node [input, name=input] {};
    \node [block, right= of input] (pre1) {Decode};
    \node [block, right= of pre1] (analysis1) {Track};
    \node [block, right= of analysis1] (analysis2) {Compare};
    \node [block, right= of analysis2] (analysis3) {FilterErrors};
    \node [block, right= of analysis3] (analysis4) {Aggregate};
    \node [block, right= of analysis4] (stabi) {Transform \textit{(Stabilize)}};
    \node [block, right= of stabi] (post1) {Reframe};
    \node [block, right= of post1] (post2) {Encode};
    \node [input, right=of post2] (output) {t};

    \currentSD

    % \node [storage, below right=0.5cm and 0.7cm of analysis] (analysisstorage) {DB};

     \draw[->](input) -- node {Video}(pre1);
     \draw[->](pre1) -- node[label] {Frames} node[below]{$f(n)$}(analysis1);
     \draw[->](analysis1) -- node[label]{Motiondata} node[below]{$m(n)$}(analysis2);
     \draw[->](analysis2) -- node[label]{Delta Tracking} node[below]{$d(n, n+1)$}(analysis3);
     \draw[->](analysis3) -- node[label,above]{Delta Tracking} node[below]{$d(n, n+1)$}(analysis4);
     \draw[->](analysis4) -- node[label,above]{Global Delta Tracking} node[below]{$gd(n, n+1)$}(stabi);
     \draw[->](stabi) -- node[label,above]{Frames} node[below]{$f(n)$}(post1);
     \draw[->](post1) -- node[label,above]{Frames} node[below]{$f(n)$}(post2);

     \draw[->](post2) -- node[above]{Video}(output);
\end{tikzpicture}
}
\end{textblock*}

\end{frame}

\begin{frame}
\frametitle{Bewegungsmuster}
    \begin{figure}\centering
        \begin{minipage}{.45\textwidth}\centering
            \begin{tikzpicture}[scale=1]
            \draw[step=0.5cm,gray,thin] (-3,-2) grid (2,1);
            \draw[black, fill = black] (-2,-0.5) circle [radius=.5];
            \draw[black, pattern=dots] (1,-0.5) circle [radius=.5];
            \draw[thick, black, ->, line width=1mm] (-1.25,-0.5) -- node[above]{\small$\overrightarrow{v} = (6,0)$} (0.25,-0.5);
            \end{tikzpicture}
            \subcaption{Translation}
        \end{minipage}
        \begin{minipage}{.45\textwidth}\centering
            \begin{tikzpicture}[scale=1]
            \draw[step=0.5cm,gray,thin] (-3,-2) grid (2,1);

            \draw[black, fill = black]  (-2.5,0) rectangle (-1.5,-1);
            \draw[black, pattern=dots,rotate around={45:(1,-0.5)}] (0.5,0) rectangle (1.5,-1);
            % \draw[thick, black, ->, line width=1mm] (-1.25,-0.5) -- (0.15,-0.5);
            \draw[-stealth,  black, line width=0.5mm] (-1.75,0.25) arc  (100:0:0.5)node[above right]{\small$45^\circ$};
            \end{tikzpicture}
            \subcaption{Rotation}
        \end{minipage}
        % \caption{Zwei }
        \label{fig:motionmodels}
    \end{figure}
    % SMALL SD
    \renewcommand{\currentSD}{\draw[red,thick,dotted] ($(analysis1.north west)+(-0.2,0.2)$)  rectangle ($(analysis2.south east)+(0.2,-0.2)$);}
    

\begin{textblock*}{\the\textwidth}(30pt,215pt)\centering
\resizebox{\textwidth}{!}{%
\begin{tikzpicture}[node distance=2cm and 2.55cm,auto]\centering
    \node [input, name=input] {};
    \node [block, right= of input] (pre1) {Decode};
    \node [block, right= of pre1] (analysis1) {Track};
    \node [block, right= of analysis1] (analysis2) {Compare};
    \node [block, right= of analysis2] (analysis3) {FilterErrors};
    \node [block, right= of analysis3] (analysis4) {Aggregate};
    \node [block, right= of analysis4] (stabi) {Transform \textit{(Stabilize)}};
    \node [block, right= of stabi] (post1) {Reframe};
    \node [block, right= of post1] (post2) {Encode};
    \node [input, right=of post2] (output) {t};

    \currentSD

    % \node [storage, below right=0.5cm and 0.7cm of analysis] (analysisstorage) {DB};

     \draw[->](input) -- node {Video}(pre1);
     \draw[->](pre1) -- node[label] {Frames} node[below]{$f(n)$}(analysis1);
     \draw[->](analysis1) -- node[label]{Motiondata} node[below]{$m(n)$}(analysis2);
     \draw[->](analysis2) -- node[label]{Delta Tracking} node[below]{$d(n, n+1)$}(analysis3);
     \draw[->](analysis3) -- node[label,above]{Delta Tracking} node[below]{$d(n, n+1)$}(analysis4);
     \draw[->](analysis4) -- node[label,above]{Global Delta Tracking} node[below]{$gd(n, n+1)$}(stabi);
     \draw[->](stabi) -- node[label,above]{Frames} node[below]{$f(n)$}(post1);
     \draw[->](post1) -- node[label,above]{Frames} node[below]{$f(n)$}(post2);

     \draw[->](post2) -- node[above]{Video}(output);
\end{tikzpicture}
}
\end{textblock*}

\end{frame}
\note[itemize]{
\item Perspektive fehlt: Kann man aber implementieren ueber die Groesse des Tracking Windows und abhangigkeiten von den TVs: Inverse ist berechnbar
}

\begin{frame}
\frametitle{Bewegungserkennung}
    \begin{description}
        \item[Lokal] $\Delta(trackingVector(F(n-1)), trackingVectorF(n))$
        \item[Global] ~\\
            \textbf{1.} Ähnlichkeitmass berechnen\\
            \textbf{2.} Ueberpruefen ob genug Vekoren / Winkel aehnlich genug sind\\
            \textbf{3.} Durchschnittswert der aehnlichen Faktoren berechnen und daraus Inverse bilden

    \end{description}
    % SMALL SD
    \renewcommand{\currentSD}{\draw[red,thick,dotted] ($(analysis2.north west)+(-0.2,0.4)$)  rectangle ($(analysis4.south east)+(0.2,-0.4)$);}
    

\begin{textblock*}{\the\textwidth}(30pt,215pt)\centering
\resizebox{\textwidth}{!}{%
\begin{tikzpicture}[node distance=2cm and 2.55cm,auto]\centering
    \node [input, name=input] {};
    \node [block, right= of input] (pre1) {Decode};
    \node [block, right= of pre1] (analysis1) {Track};
    \node [block, right= of analysis1] (analysis2) {Compare};
    \node [block, right= of analysis2] (analysis3) {FilterErrors};
    \node [block, right= of analysis3] (analysis4) {Aggregate};
    \node [block, right= of analysis4] (stabi) {Transform \textit{(Stabilize)}};
    \node [block, right= of stabi] (post1) {Reframe};
    \node [block, right= of post1] (post2) {Encode};
    \node [input, right=of post2] (output) {t};

    \currentSD

    % \node [storage, below right=0.5cm and 0.7cm of analysis] (analysisstorage) {DB};

     \draw[->](input) -- node {Video}(pre1);
     \draw[->](pre1) -- node[label] {Frames} node[below]{$f(n)$}(analysis1);
     \draw[->](analysis1) -- node[label]{Motiondata} node[below]{$m(n)$}(analysis2);
     \draw[->](analysis2) -- node[label]{Delta Tracking} node[below]{$d(n, n+1)$}(analysis3);
     \draw[->](analysis3) -- node[label,above]{Delta Tracking} node[below]{$d(n, n+1)$}(analysis4);
     \draw[->](analysis4) -- node[label,above]{Global Delta Tracking} node[below]{$gd(n, n+1)$}(stabi);
     \draw[->](stabi) -- node[label,above]{Frames} node[below]{$f(n)$}(post1);
     \draw[->](post1) -- node[label,above]{Frames} node[below]{$f(n)$}(post2);

     \draw[->](post2) -- node[above]{Video}(output);
\end{tikzpicture}
}
\end{textblock*}

\end{frame}


\begin{frame}
\frametitle{Ähnlichkeit}
    \begin{description}
        \item[Vektoren] Kosinus-Ähnlichkeit
        \[cos\_sim = \frac{A \cdot B}{\Vert A\Vert \Vert B\Vert} = \frac{\sum\limits_{i=1}^{n} A_iB_i}{\sqrt{\sum\limits_{i=1}^{n}A_i^2} \sqrt{\sum\limits_{i=1}^{n}B_i^2}}\]
        \item[Winkel] Prozentuale Veraenderung
    \end{description}
    % SMALL SD
    \renewcommand{\currentSD}{\draw[red,thick,dotted] ($(analysis4.north west)+(-0.2,0.2)$)  rectangle ($(analysis4.south east)+(0.2,-0.2)$);}
    

\begin{textblock*}{\the\textwidth}(30pt,215pt)\centering
\resizebox{\textwidth}{!}{%
\begin{tikzpicture}[node distance=2cm and 2.55cm,auto]\centering
    \node [input, name=input] {};
    \node [block, right= of input] (pre1) {Decode};
    \node [block, right= of pre1] (analysis1) {Track};
    \node [block, right= of analysis1] (analysis2) {Compare};
    \node [block, right= of analysis2] (analysis3) {FilterErrors};
    \node [block, right= of analysis3] (analysis4) {Aggregate};
    \node [block, right= of analysis4] (stabi) {Transform \textit{(Stabilize)}};
    \node [block, right= of stabi] (post1) {Reframe};
    \node [block, right= of post1] (post2) {Encode};
    \node [input, right=of post2] (output) {t};

    \currentSD

    % \node [storage, below right=0.5cm and 0.7cm of analysis] (analysisstorage) {DB};

     \draw[->](input) -- node {Video}(pre1);
     \draw[->](pre1) -- node[label] {Frames} node[below]{$f(n)$}(analysis1);
     \draw[->](analysis1) -- node[label]{Motiondata} node[below]{$m(n)$}(analysis2);
     \draw[->](analysis2) -- node[label]{Delta Tracking} node[below]{$d(n, n+1)$}(analysis3);
     \draw[->](analysis3) -- node[label,above]{Delta Tracking} node[below]{$d(n, n+1)$}(analysis4);
     \draw[->](analysis4) -- node[label,above]{Global Delta Tracking} node[below]{$gd(n, n+1)$}(stabi);
     \draw[->](stabi) -- node[label,above]{Frames} node[below]{$f(n)$}(post1);
     \draw[->](post1) -- node[label,above]{Frames} node[below]{$f(n)$}(post2);

     \draw[->](post2) -- node[above]{Video}(output);
\end{tikzpicture}
}
\end{textblock*}

\end{frame}
\note[itemize]{
    \item Kosinus-Ähnlichkeit ist ein Maß für die Ähnlichkeit zweier Vektoren
    \item Kosinus des Winkels zwischen beiden Vektoren
    \item 1:  Abhängigkeit // 0: Unabhängigkeit // -1: Exakt gegenteil
    \item $[-1\leq cos\_sim \leq,1]$
    \item  Attribut-Vektoren a {\displaystyle \mathbf {a} } \mathbf {a} und b {\displaystyle \mathbf {b} } \mathbf{b}
    \item Der Kosinus zweier Vektoren bestimmt sich aus dem Standardskalarprodukt: $a\cdot b = \Vert A\Vert\Vert B\Vert cos \Theta$\\[1.5em]
    \item $A \cdot B$ -> Skalarprodukt
    \item $A \cdot B$ -> Norm:  Abbildung die dem Vektor eine Zahl zuordnet, die auf gewisse Weise die Größe des Objekts beschreiben soll
}

\begin{frame}
\frametitle{Stabilisierung}
    \begin{figure}\centering
        \begin{minipage}{.45\textwidth}\centering
            \begin{tikzpicture}[scale=1]
            \draw[step=0.5cm,gray,thin] (-3,-2) grid (2,1);
            \draw[black, fill = black] (-2,-0.5) circle [radius=.5];
            \draw[black, pattern=dots] (1,-0.5) circle [radius=.5];
            \draw[thick, black, ->, line width=1mm] (-1.25,-0.5) -- node[above]{\small$\overrightarrow{v} = (6,0)$} (0.25,-0.5);
            \draw[thick, white, ->, line width=0.001mm] (-1.25,1.1) -- node[above]{\small$\overrightarrow{v} = (6,0)$} (-2.75,1.1);
            \end{tikzpicture}
            \subcaption{Erkennung}
        \end{minipage}
        \begin{minipage}{.45\textwidth}\centering
            \begin{tikzpicture}[scale=1]
            \draw[step=0.5cm,gray,thin] (-3,-2) grid (2,1);
            \draw[black, fill = gray] (-2,-0.5) circle [radius=.5];
            \draw[thick, blue, ->, line width=0.5mm] (-1.25,1.1) -- node[above]{\small$\overrightarrow{v}_{inverse} = (-6,0)$} (-2.75,1.1);

            \draw[step=0.5cm,blue,thin] (-5,-2) grid (-1,1);
            \draw[blue, pattern=dots] (-2,-0.5) circle [radius=.5];


            \end{tikzpicture}
            \subcaption{Inverse}
        \end{minipage}
        \caption{Translationsstabilisierung}
        \label{fig:motionmodels}
    \end{figure}
    % SMALL SD
    \renewcommand{\currentSD}{\draw[red,thick,dotted] ($(stabi.north west)+(-0.2,0.2)$)  rectangle ($(stabi.south east)+(0.2,-0.2)$);}
    

\begin{textblock*}{\the\textwidth}(30pt,215pt)\centering
\resizebox{\textwidth}{!}{%
\begin{tikzpicture}[node distance=2cm and 2.55cm,auto]\centering
    \node [input, name=input] {};
    \node [block, right= of input] (pre1) {Decode};
    \node [block, right= of pre1] (analysis1) {Track};
    \node [block, right= of analysis1] (analysis2) {Compare};
    \node [block, right= of analysis2] (analysis3) {FilterErrors};
    \node [block, right= of analysis3] (analysis4) {Aggregate};
    \node [block, right= of analysis4] (stabi) {Transform \textit{(Stabilize)}};
    \node [block, right= of stabi] (post1) {Reframe};
    \node [block, right= of post1] (post2) {Encode};
    \node [input, right=of post2] (output) {t};

    \currentSD

    % \node [storage, below right=0.5cm and 0.7cm of analysis] (analysisstorage) {DB};

     \draw[->](input) -- node {Video}(pre1);
     \draw[->](pre1) -- node[label] {Frames} node[below]{$f(n)$}(analysis1);
     \draw[->](analysis1) -- node[label]{Motiondata} node[below]{$m(n)$}(analysis2);
     \draw[->](analysis2) -- node[label]{Delta Tracking} node[below]{$d(n, n+1)$}(analysis3);
     \draw[->](analysis3) -- node[label,above]{Delta Tracking} node[below]{$d(n, n+1)$}(analysis4);
     \draw[->](analysis4) -- node[label,above]{Global Delta Tracking} node[below]{$gd(n, n+1)$}(stabi);
     \draw[->](stabi) -- node[label,above]{Frames} node[below]{$f(n)$}(post1);
     \draw[->](post1) -- node[label,above]{Frames} node[below]{$f(n)$}(post2);

     \draw[->](post2) -- node[above]{Video}(output);
\end{tikzpicture}
}
\end{textblock*}

\end{frame}
%
\begin{frame}
\frametitle{Stabilisierung}
\vspace{-1.2em}
    \begin{figure}\centering
        \begin{minipage}{.4\textwidth}\centering
            \begin{tikzpicture}[scale=1]
            \draw[step=0.5cm,gray,thin] (-3,-2) grid (2,1);

            \draw[black, fill = black]  (-2.5,0) rectangle (-1.5,-1);
            \draw[black, pattern=dots,rotate around={45:(1,-0.5)}] (0.5,0) rectangle (1.5,-1);
            % \draw[thick, black, ->, line width=1mm] (-1.25,-0.5) -- (0.15,-0.5);
            \draw[-stealth,  black, line width=0.5mm] (-1.75,0.25) arc  (100:0:0.5)node[above right]{\small$45^\circ$};
            \end{tikzpicture}
            \subcaption{Erkennung}
        \end{minipage}
        \begin{minipage}{.58\textwidth}\centering
            \begin{tikzpicture}[scale=1]
            \draw[step=0.5cm,gray,thin] (-3,-2) grid (2,1);
            \draw[step=0.5cm,blue,rotate around={45:(1,-0.5)},thin] (-3,-2) grid (2,1);

            \draw[black, fill = black]  (-2.5,0) rectangle (-1.5,-1);
            \draw[blue, pattern=dots,] (0.5,0) rectangle (1.5,-1);
            % \draw[thick, black, ->, line width=1mm] (-1.25,-0.5) -- (0.15,-0.5);
            \draw[-stealth,  blue, line width=0.5mm] (-2.7,-1.8) arc  (80:275:0.5)node[below left]{\small$-45^\circ$};
            \end{tikzpicture}
            \subcaption{Inverse}
        \end{minipage}
% \protect\caption[position=left]{Rotationsstabilisierung}
        % {\hspace*{-2em}\caption{Rotationsstabilisierung}}
        \label{fig:motionmodels}
    \end{figure}
    % SMALL SD
    \renewcommand{\currentSD}{\draw[red,thick,dotted] ($(stabi.north west)+(-0.2,0.2)$)  rectangle ($(stabi.south east)+(0.2,-0.2)$);}
    

\begin{textblock*}{\the\textwidth}(30pt,215pt)\centering
\resizebox{\textwidth}{!}{%
\begin{tikzpicture}[node distance=2cm and 2.55cm,auto]\centering
    \node [input, name=input] {};
    \node [block, right= of input] (pre1) {Decode};
    \node [block, right= of pre1] (analysis1) {Track};
    \node [block, right= of analysis1] (analysis2) {Compare};
    \node [block, right= of analysis2] (analysis3) {FilterErrors};
    \node [block, right= of analysis3] (analysis4) {Aggregate};
    \node [block, right= of analysis4] (stabi) {Transform \textit{(Stabilize)}};
    \node [block, right= of stabi] (post1) {Reframe};
    \node [block, right= of post1] (post2) {Encode};
    \node [input, right=of post2] (output) {t};

    \currentSD

    % \node [storage, below right=0.5cm and 0.7cm of analysis] (analysisstorage) {DB};

     \draw[->](input) -- node {Video}(pre1);
     \draw[->](pre1) -- node[label] {Frames} node[below]{$f(n)$}(analysis1);
     \draw[->](analysis1) -- node[label]{Motiondata} node[below]{$m(n)$}(analysis2);
     \draw[->](analysis2) -- node[label]{Delta Tracking} node[below]{$d(n, n+1)$}(analysis3);
     \draw[->](analysis3) -- node[label,above]{Delta Tracking} node[below]{$d(n, n+1)$}(analysis4);
     \draw[->](analysis4) -- node[label,above]{Global Delta Tracking} node[below]{$gd(n, n+1)$}(stabi);
     \draw[->](stabi) -- node[label,above]{Frames} node[below]{$f(n)$}(post1);
     \draw[->](post1) -- node[label,above]{Frames} node[below]{$f(n)$}(post2);

     \draw[->](post2) -- node[above]{Video}(output);
\end{tikzpicture}
}
\end{textblock*}

\end{frame}


\begin{frame}
\frametitle{Konzepte}
    \begin{columns}
    \begin{column}{0.32\textwidth}\centering
        test
    \end{column}
    \begin{column}{0.32\textwidth}\centering
        test
    \end{column}
    \begin{column}{0.32\textwidth}\centering
        test
    \end{column}
    \end{columns}
\end{frame}

\begin{frame}
\frametitle{Konzeptdefinition}
\end{frame}

\begin{frame}
\frametitle{Implementierung}
\end{frame}

\begin{frame}
\frametitle{Zukunft}
    \begin{description}
        \item[Smoothing] Die Stabilisierung ist aktuell noch sehr harsch. Dies kann durch ein \textit{Smoothing} der globalen Delta Vektoren optimiert werden.
        \item[Reframing] Es entstehen aktuell noch schwarze Raender, die bei der Stabilisierung genutzt werden. Druch eine Analyse welche Pixel immer belegt sind, kann das Video beschnitten werden, um eine bessere Nutzererfahrung zu bieten. Moeglich ist es auch, die enstehenden Raender automatisch mit Pixeln zu ergaenzen, um ein stimmiges Gesamtbild ohne \textit{Crop} zu erzeugen \cite{blockTang}.
    \end{description}
\end{frame}

\begin{frame}%[allowframebreaks]
\frametitle{Referenzen}
    \printbibliography[heading=none]
\end{frame}

\begin{frame}
\begin{center}
    \textbf{\huge BACKUP}
\end{center}
\end{frame}

\begin{frame}
    \frametitle{CamShift}
    \begin{figure}[h!]
        \resizebox{\textwidth}{!}{%
        
    \tcbox[enhanced, sharp corners, boxsep=1mm, boxrule=1mm,colframe=gray!10!white,
        colback=white,
        borderline={2pt}{-1pt}{gray,dotted}]
        {\begin{tikzpicture}[scale=0.6,node distance=1cm and 2.5cm,auto]
            \node [input, name=input] {};
            \node [block, right= of input] (pre1) {Initialize tracking areas};
            \node [wideblock, right= of pre1] (pre2) {Histogram \& Backprojection};
            \node [block, right= of pre2] (pre3) {Normali-sation};
            \node [block, below= of pre3] (pre4) {Mean-Shift};
            \node [wideblock, left= of pre4] (pre5) {Adjust size of tacking area and rotation};
            \node [input, left=of pre5] (output) {t};

            \draw[->](input) -- node[label]{Frames}node[below]{$f(n)$}(pre1);
            \draw[->](pre1) -- node[label]{Frames}node[below]{$f(n)$}(pre2);


            \draw[->](pre2) -- node[label]{$bj(f(n))$}node[below]{$hsv(f(n))$}(pre3);

            \draw[->](pre3) -- node[label, left]{$bj(f(n))$}node[right]{$hsv(f(n))$}(pre4);


            \draw[->](pre4) -- node[label, above]{Tackwindow}node[below]{$tv(f(n), ta)$}(pre5);
            \draw[->](pre5)  |- ++(0,-3cm) --node[label, above]{Tackwindow}node[below]{$tv(f(n), ta)$} ++(8.8cm,0) -- (pre4);


            \draw[->](pre5) -- node[label, above]{Motiondata}node[below]{$m(n)$}(output);
        \end{tikzpicture}
        }
}
        \caption{Detailed system diagram of \textit{Track}}
    \end{figure}
\end{frame}

\end{document}
