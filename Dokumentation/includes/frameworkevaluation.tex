This section starts with an describtion of the scoring criterias which will be used in \cref{sec:evaluation} to evaluate the in \cref{sec:frameworks} described frameworks. The final decision with a detailied explanation is closing this section.

\subsection{Evaluation criteria}\label{sec:requirementanal}
The overall goal of this work is to develop a \texttt{C++} driven software video stabilisation. Therefore, the following criteria are the basis to evaluate the perfect \texttt{C++} framework with all needed features to extract, analyse and manipulate videos.

\begin{description}
    \item[Videocodecs] The framework needs to be able to handle basic codecs. In detail these are \tbd
    \item[Extracting frames \textit{Decoding}] One basic feature of a suitable framework is the extraction of every frame to process it further.
    \item[Edge detection] One approach to the problem is to recognize edges and to track their movement. Therefore, a suitable framework needs to be able to extract edges and to present them in an efficent way.
    \item[Manipulation of frames] A suitable framework needs to be able to transform, rotate and warp frames.
    \item[Tracking] An non-necessary but useful feature is the object tracking. These objects could be used to stabilize videos around an important object.
    % \item[]
    \item[Merge frames \textit{Encoding}] Another basic feature of a suitable framework is to merge the manipulated frames back into an video.
\end{description}

\subsection{Frameworks}\label{sec:frameworks}
\subsection{Evaluation}\label{sec:evaluation}
\subsection{Decision}\label{sec:decision}
